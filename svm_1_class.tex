\documentclass[a4paper, parskip=full]{scrartcl}
%\documentclass[parskip=full]{scrartcl}
\usepackage[margin=0.9in]{geometry}
\usepackage[utf8]{inputenc}
\usepackage{amsmath}
\usepackage{amsfonts}
\usepackage{amssymb}
\usepackage{graphicx}
\usepackage{caption}
\usepackage{subcaption}
\usepackage{listings}
\usepackage{color}
\usepackage[colorinlistoftodos]{todonotes}

\definecolor{dkgreen}{rgb}{0,0.6,0}
\definecolor{gray}{rgb}{0.5,0.5,0.5}
\definecolor{mauve}{rgb}{0.58,0,0.82}

\lstset{frame=tb,
	language=Python,
	aboveskip=3mm,
	belowskip=3mm,
	showstringspaces=false,
	columns=flexible,
	basicstyle={\small\ttfamily},
	numbers=none,
	numberstyle=\tiny\color{gray},
	keywordstyle=\color{blue},
	commentstyle=\color{dkgreen},
	stringstyle=\color{mauve},
	breaklines=true,
	breakatwhitespace=true,
	tabsize=3
}
\date{}
\usepackage{authblk}
\graphicspath{{/home/suraj/Desktop/Novin/images/}}
	



\begin{document}
	

\begin{titlepage}
	
	\newcommand{\HRule}{\rule{\linewidth}{0.5mm}} % Defines a new command for the horizontal lines, change thickness here
	
	\center % Center everything on the page
	
	%----------------------------------------------------------------------------------------
	%	HEADING SECTIONS
	%----------------------------------------------------------------------------------------
	
%	\textsc{\LARGE University Name}\\[1.5cm] % Name of your university/college
%	\textsc{\Large Major Heading}\\[0.5cm] % Major heading such as course name
%	\textsc{\large Minor Heading}\\[0.5cm] % Minor heading such as course title
	
	%----------------------------------------------------------------------------------------
	%	TITLE SECTION
	%----------------------------------------------------------------------------------------
	
	\HRule \\[0.4cm]
	{ \huge \bfseries Developing Anomaly detection for elderly people Oriented IoT Devices}\\[0.4cm] % Title of your document
	\HRule \\[1.5cm]
	
	%----------------------------------------------------------------------------------------
	%	AUTHOR SECTION
	%----------------------------------------------------------------------------------------
	
	\begin{minipage}{0.4\textwidth}
		\begin{flushleft} \large
			\emph{Author:}\\
			Surajudeen \textsc{Abdulrasaq} % Your name
		\end{flushleft}
	\end{minipage}
	~
	\begin{minipage}{0.4\textwidth}
		\begin{flushright} \large
			\emph{Supervisor:} \\
			Jérémie  \textsc{BENNEGENT} % Supervisor's Name
		\end{flushright}
	\end{minipage}\\[2cm]
	
	% If you don't want a supervisor, uncomment the two lines below and remove the section above
%	\Large \emph{Author:}\\
%	John \textsc{Smith}\\[3cm] % Your name
	
	%----------------------------------------------------------------------------------------
	%	DATE SECTION
	%----------------------------------------------------------------------------------------
	
	{\large \ Eight report submitted on 28-May-2018}\\[0.2cm] % Date, change the \today to a set date if you want to be precise
	
	%----------------------------------------------------------------------------------------
	%	LOGO SECTION
	%----------------------------------------------------------------------------------------
	
	\includegraphics{orange-cane3.png}\\[0.2cm] % Include a department/university logo - this will require the graphicx package
	
	%----------------------------------------------------------------------------------------
	
	\vfill % Fill the rest of the page with whitespace
	
\end{titlepage}







 \section*{One-Class SVM}
 
One-Class SVM is a special case of Support Vector Machine that learn a hyper plane by separating all the data points from the basis and constructs a smooth boundary around the majority of probability mass of data, it is an unsupervised algorithm that learns a decision function for uniqueness detection by classifying new data as similar or dissimilar to the training set which makes it suitable for detecting anomaly.

given a  data-set $X$, with an unknown label, and a $\Phi(X)$  RKHS map (kernel Hilbert space) function from the input space to the feature space $F$, a decision function $f(X_n)$ in the feature space $F$ is given as $f(X_n) = w^T\Phi(X_n) -r$ , to separate as many as possible of the mapped vectors $\Phi(X n: ),n : 1, 2, ...,N$ from the origin. Where $w$ is the norm perpendicular to the hyper-plane and $r$ represent the bias of the hyper-plane, then we arrive at:

\[min_{(w,r)}\frac{1}{2}\|w\|_2^2 + \dfrac{1}{v}\ldotp \dfrac{1}{N}\sum_{i =1}^{n}max (0,r -\langle w,\Phi(X_n:)\rangle) - r \]

$v$ is the is the parameter that represent the balance between maximizing the hyper-plane and the total data-point permitted across the boundary and $v$ ranges between (0, 1).


\newpage

\begin{figure}
	
	\centering
	\includegraphics[width=\linewidth]{anomaly_svm.png}
	\caption{Anomaly Detection with One-Class SVM }
	
\end{figure}

\begin{figure}
	\centering
	\includegraphics[width=\linewidth]{new_test_data_SVM.png}
	\caption{ Shows One-Class SVM with new test-data (note the anomaly in white).}
	
\end{figure}

\end{document}